\section{The \Queso\ input file}
\begin{frame}[fragile]{\Queso\ input file: Environment options}
  \begin{itemize}
    \item Already seen how to change number of chains with input file
    \item Prefix stuff is here.  \texttt{env\_} is the default prefix is empty
      for the \texttt{FullEnvironment}.  You can change it and the
      input file options will need to be prefixed with the passed user prefix.
    \item Some environment options:
      \begin{itemize}
        \item \texttt{env\_subDisplayFileName} is the place to \Queso\ will
          put general diagnostic information
        \item \texttt{env\_subDisplayAllowAll} toggles whether or not all
          process can write output related to environment
        \item \texttt{env\_subDisplayAllowedSet} if the above option is false,
          this option allows you to specify a subset of processes that can
          write output
        \item \texttt{env\_seed} is a flag for setting random number generator
          seed.  Set this to something negative if you want multiple concurrent
          chains to produce distinct samples.  If nonzero, this is the seed
          that is used for all chains.
      \end{itemize}
  \end{itemize}
\end{frame}

\begin{frame}[fragile]{\Queso\ input file: Inverse problem options}
  \begin{itemize}
    \item Inverse problem options
      \begin{itemize}
        \item \texttt{ip\_computeSolution} is a boolean.  If false, no
          computation of the inverse problem is done.  Useful for testing
          setup.
        \item \texttt{ip\_dataOutputFileName} is the output file for inverse
          problem related information.
        \item \texttt{ip\_dataOutputAllowedSet} similar to environment option,
          but specific to inverse problem
      \end{itemize}
  \end{itemize}
\end{frame}

\begin{frame}[fragile]{\Queso\ input file: Algorithm options}
  \begin{itemize}
    \item All DRAM options are documented as part of the
      \texttt{MhOptionsValues} class
      \begin{itemize}
        \item \url{http://libqueso.com/queso/html/a00166.html}
      \end{itemize}
    \item Software documentation and usability is assessed by users
    \item If there are any options (anywhere, not just DRAM) that are unclear
      or undocumented feel free to open a new GitHub ticket
    \item Better yet, write a patch to fix it and you can be a \Queso\
      contributor!
  \end{itemize}
\end{frame}

\begin{frame}[fragile]{Programmatically setting options}
  \begin{itemize}
    \item You can set options programmatically instead of via the input file
    \item Objects whose behaviour is tweakable will accept a pointer to an
      instance of \texttt{*OptionsValues} in the constructor
      \begin{itemize}
        \item \texttt{FullEnvironment} takes an \texttt{EnvOptionsValues}
          pointer: \url{http://libqueso.com/queso/html/a00081.html#a77081a9fd8cb7b90ee3c0da289d91815}
        \item \texttt{StatisticalInverseProblem} takes an
          \texttt{SipOptionsValues} pointer: \url{http://libqueso.com/queso/html/a00207.html#a98ad98a7030b2c6577a0e840506bf74d}
        \item Options object member names are similar to input file option names
      \end{itemize}
    \item You needn't create a \texttt{MetropolisHastingsSG} object; the
      inverse problem will create one for you
    \item You can pass options via \texttt{solveWithBayesMetropolisHastings}:
      \begin{itemize}
        \item \url{http://libqueso.com/queso/html/a00207.html#a924189e647110129682308b9bffc3a0d}
      \end{itemize}
  \end{itemize}
\end{frame}

\begin{frame}[fragile]{Thinning}
  \begin{itemize}
    \item A common way to try and improve decorrelation time is by thinning
    \item Thinning is simply saving every $m$-th sample where $m > 1$
    \item The \texttt{ip\_raw\_chain.m} contains the unthinned chain
    \item To enable thinning, turn on this input file option:
      \begin{itemize}
        \item \texttt{ip\_mh\_filteredChain\_generate = 1}
      \end{itemize}
    \item With this option on, you must also set the value for $m$:
      \begin{itemize}
        \item \texttt{ip\_mh\_filteredChain\_lag = 2}
      \end{itemize}
    \item Somewhere to store the output:
      \begin{itemize}
        \item \texttt{ip\_mh\_filteredChain\_dataOutputFileName = outputData/ip\_filtered\_chain}
      \end{itemize}
    \item Let all processes write their output
      \begin{itemize}
        \item \texttt{ip\_mh\_filteredChain\_dataOutputAllowAll = 1}
      \end{itemize}
  \end{itemize}
\end{frame}

\section{Task 6}
\begin{frame}[fragile]{Task 6}
  \begin{itemize}
    \item Enable thinning
    \item Perhaps remove (or backup) the existing \texttt{outputData} directory
    \item Run the code (perhaps with multiple chains in parallel)
    \item Verify the length of the filtered chain is what you expect
    \item Compute the running mean of the samples.  Does it converge faster
      or slower?
    \item This task should take less than twenty minutes.
  \end{itemize}
\end{frame}
