\section{Matrices}
\begin{frame}[fragile]{Matrices}
  \begin{itemize}
    \item All matrices are square.
    \item Obviously not true, but most of the time we only care about square
      (covariance) matrices.
    \item You can create nonsquare matrices, but I doubt you will ever need
      one in \Queso\ unless your forward problem needs it.
    \item Matrices don't \emph{live} in the vector space, but they are defined
      over it.
    \item Use an existing \texttt{GslVector} to construct a square matrix with
      compatible dimension.
  \end{itemize}
  \begin{verbatim}
QUESO:GslMatrix myMatrix(myVector);
  \end{verbatim}
  \begin{itemize}
    \item Creates diagonal matrix with element $(i, i)$ equal to
      \texttt{myVector[i]}.
    \item Argument is a \texttt{GslVector} object but could be another
      \texttt{GslMatrix}, in which case the whole matrix is copied.
  \end{itemize}
\end{frame}

\begin{frame}[fragile]{Matrices}
  Set element $(3, 4)$:
  \begin{verbatim}
myMatrix(2, 3) = 1.0;
  \end{verbatim}
  Get element $(2, 3)$:
  \begin{verbatim}
double myElement = myMatrix(1, 2);
  \end{verbatim}
  \begin{itemize}
    \item \Queso\ has methods that can factorise matrices in order to solve
      linear systems.
    \item Linear algebra all relies on \Gsl.
    \item See above comment about ongoing work for high-performance
      vector/matrix implementations.
    \item For other methods, see
      \url{http://libqueso.com/queso/html/a00127.html}
  \end{itemize}
\end{frame}

\section{Task 2}
\begin{frame}[fragile]{Task 2}
  \begin{itemize}
    \item Create a \texttt{GslVector}.
    \item Create a \texttt{GslMatrix}.
    \item Multiply the vector by the matrix.
    \item Verify the result of the multiplication is correct.
    \item If you complete this quickly, look at the documentation and play with
      some of the other methods.
    \item This task should take less than ten minutes.
  \end{itemize}
\end{frame}
