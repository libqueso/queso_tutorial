\section{The environment}
\begin{frame}[fragile]{The environment}
  \begin{itemize}
    \item All classes are available in the \texttt{QUESO} namespace.
    \item Most of the objects in \Queso\ need an \emph{environment}.
    \item Some objects can infer it from other objects.
    \item Lives in \texttt{queso/Enivornment.h} and called \texttt{FullEnvironment}.
  \end{itemize}
  \begin{verbatim}
#include <queso/Environment.h>

int main(int argc, char ** argv)
{
  QUESO::FullEnvironment env(MPI_COMM_WORLD, argv[1], "",
                             NULL);
  return 0;
}
  \end{verbatim}
\end{frame}

\begin{frame}[fragile]{The environment}
  \begin{verbatim}
QUESO::FullEnvironment env(MPI_COMM_WORLD, argv[1], "",
                           NULL);
  \end{verbatim}
  Arguments:
  \begin{enumerate}
    \item \Mpi\ communitcator.
    \item Path to \Queso\ input file.
    \item Prefix string for entries in the input file.
    \item Optional programatically provided input options.
  \end{enumerate}
\end{frame}

\section{Task 1}
\begin{frame}[fragile]{Task 1}
  \begin{itemize}
    \item Create a source file and instantiate a \texttt{FullEnvironment}.
    \item Compile it, link it against \Queso\ and run it.
    \item  Make sure there are no errors.
    \item You might need an input file.  You can find one here:
      \url{https://github.com/libqueso/queso/blob/dev/examples/template_example/template_example_input.txt}
    \item We'll talk about the input file later.  For now you can take it for
      granted.
    \item This task should take less than five minutes.
  \end{itemize}
\end{frame}
