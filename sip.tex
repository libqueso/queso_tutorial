\section{Statistical inverse problem}
\begin{frame}[fragile]{Statistical inverse problem}
  \begin{itemize}
    \item So far, have created a prior
    \item Have created a likelihood
    \item We will now from the statistical inverse problem
    \item \Queso\ will solve the statistical inverse problem
  \end{itemize}
\end{frame}

\begin{frame}[fragile]{Statistical inverse problem}
  Create a posterior random variable (on the same domain):
  \begin{verbatim}
QUESO::GenericVectorRV<> postRv("post_", paramSpace); \end{verbatim}
  Create a statistical inverse problem with the prior, likelihood and
  posterior:
  \begin{verbatim}
QUESO::StatisticalInverseProblem<> ip("", NULL,
                                      priorRv,
                                      lhood, postRv); \end{verbatim}
\end{frame}

\begin{frame}[fragile]{Statistical inverse problem}
  Before we solve the problem we have to tell \Queso\ what the first sample in
  the chain is:
  \begin{verbatim}
QUESO::GslVector paramInitials(paramSpace.zeroVector());
paramInitials[0] = 1.0;
paramInitials[1] = 1.0; \end{verbatim}
  And we also have to tell \Queso\ the variance of the proposal distribution:
  \begin{verbatim}
QUESO::GslMatrix propCovMatrix(paramSpace.zeroVector());
propCovMatrix(0, 0) = 0.0001;
propCovMatrix(1, 1) = 0.0001; \end{verbatim}
  These are algorithmic knobs, and there are efforts to remove this necessity
  in future and start with sensible defaults.  For now, though, we have to set
  these.
\end{frame}

\begin{frame}[fragile]{Statistical inverse problem}
  Now solve and tidy up:
  \begin{verbatim}
  ip.solveWithBayesMetropolisHastings(NULL, paramInitials,
                                      &propCovMatrix);
  MPI_Finalize();
  return 0;
} \end{verbatim}
\end{frame}

\section{Task 5}
\begin{frame}[fragile]{Task 5}
  \begin{itemize}
    \item Compile and run the code.  Output will be in the \texttt{outputData} directory
    \item Samples are stored in the \texttt{ip\_raw\_chain\_.m} file as a
      \texttt{matlab} vector
    \item The first column are samples of the first parameter
      (\texttt{domainVector[0]}).  The second column are samples of the second
      parameter (\texttt{domainVector[1]})
    \item Matlab output is a hinderance (not open source).  There are efforts
      to output an HDF5 file in future.  Initial support exists but is not
      usable at present
    \item Use script to plot the samples to a PDF file.
    \item The truth is $(D, \beta) = (0.197, 1.75)$.
    \item Plot the running mean and variance of the samples.  What do you get?
    \item This task should take less than twenty minutes.
  \end{itemize}
\end{frame}
