\section{Random variables}
\begin{frame}[fragile]{Random variables}
  \begin{itemize}
    \item \Queso\ supports the following random variables:
      \begin{itemize}
        \item Inverse Gamma
        \item Jeffreys
        \item Log Normal
        \item Beta
        \item Gamma
        \item Gaussian
        \item Uniform
        \item Wigner
      \end{itemize}
    \item \Queso also supports creating custom random variables.
  \end{itemize}
\end{frame}

\begin{frame}[fragile]{Random variables}
  \begin{itemize}
    \item Random variables need the domain, and possibly other parameters
    \item Uniform:
      \begin{verbatim}
QUESO::UniformVectorRV<> priorRv("prior_", paramDomain); \end{verbatim}
    \item Beta needs a vector of $\alpha$ and $\beta$ parameters as well:
      \begin{verbatim}
QUESO::BetaVectorRV<> betaRv("prior_", paramDomain,
                             alphas, betas);\end{verbatim}
    \item Gaussians need a mean and covariance:
      \begin{verbatim}
QUESO::GaussianVectorRV<> gaussianRv("prior_",
                                     paramDomain, mean,
                                     cov); \end{verbatim}
    \item Unless it's obvious how to correlated elements of the vector, they
      are assumed to be independent.
  \end{itemize}
\end{frame}

\begin{frame}[fragile]{Random variables}
  \begin{itemize}
    \item \texttt{*VectorRV} classes have two major components:
      \begin{itemize}
        \item A \texttt{*JointPdf} object to evaluate the underlying density.
        \item A \texttt{*VectorRealizer} component to draw realisations.
      \end{itemize}
    \item You can access these components with the \texttt{pdf()} method\ldots
      \begin{verbatim}
QUESO::GslVector point(paramSpace.zeroVector());
betaRv.pdf().actualValue(point, NULL, NULL, NULL,
                         NULL); \end{verbatim}
    \item \ldots or the \texttt{realizer()} method
      \begin{verbatim}
QUESO::GslVector draw(paramSpace.zeroVector());
betaRv.realizer().realization(draw); \end{verbatim}
  \end{itemize}
\end{frame}

\section{Task 3}
\begin{frame}[fragile]{Task 3}
  \begin{enumerate}
    \item Create a Guassian random variable over some domain.
    \item Think about your domain for a few minutes.
    \item Compute (and print) the empirical mean and variance (see point 2.)
      from $10^6$ realisations.
    \item Are they (approximately) what you expect them to be?  If not, why
      not?
    \item This task should take less than fifteen minutes.
  \end{enumerate}
\end{frame}
